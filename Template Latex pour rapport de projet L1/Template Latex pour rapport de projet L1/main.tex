\documentclass[a4paper, 12pt, twoside]{article}
\usepackage[utf8]{inputenc}
\usepackage[T1]{fontenc}
\usepackage[francais]{babel}
\usepackage{lmodern}
\usepackage{ae,aecompl}
\usepackage[top=2.5cm, bottom=2cm,
            left=3cm, right=2.5cm,
            headheight=15pt]{geometry}
\usepackage{graphicx}
\usepackage{eso-pic}
\usepackage{array}
\usepackage{hyperref}
\input{pagedegarde}

% ====== Infos rapport À MODIFIER ======
\title{Assistant IA pour les étudiants de l'Université Paris Nanterre}
\entreprise{Université Paris Nanterre}
\datedebut{septembre 2024}
\datefin{janvier 2025}

\membrea{Nom1 Prénom1}
\membreb{Nom2 Prénom2}
% Laisse les autres vides pour respecter "exactement 2 étudiants"
\membrec{}
\membred{}
\membree{}

\begin{document}
\pagedegarde

\section*{Remerciements}
Nous remercions l'équipe enseignante de l'UE de programmation web pour l'encadrement du projet, ainsi que les personnes qui ont testé notre application et nous ont fourni des retours utiles.

\newpage
\tableofcontents
\newpage

% ================== 1. INTRODUCTION ==================
\section{Introduction}
Ce projet a été réalisé dans le cadre de l'UE de programmation web de L2 MIASHS à l'Université Paris Nanterre. 
L'objectif est de développer une application web intégrant un assistant conversationnel capable d'aider les étudiants dans leurs démarches universitaires (inscription, scolarité, vie étudiante et informations pratiques sur le campus).

L'application permet à un étudiant de se connecter, de poser des questions en langage naturel à une IA spécialisée sur le contexte de l'université et de consulter l'historique de ses échanges.

% ================== 2. ENVIRONNEMENT DE TRAVAIL ==================
\section{Environnement de travail}
Nous avons développé le projet principalement sous Linux / WSL avec l'éditeur VS Code. 
Le code est versionné avec Git et hébergé sur GitHub, ce qui a facilité le travail collaboratif et la sauvegarde de l'historique des modifications.

Le projet utilise un environnement virtuel Python (\texttt{venv}) pour isoler les dépendances (Flask, client de l'API d'IA, etc.), ainsi qu'une base de données SQLite locale pour stocker les utilisateurs et les échanges.

% ================== 3. DESCRIPTION ET OBJECTIFS ==================
\section{Description du projet et objectifs}
L'objectif principal du projet est de fournir un assistant numérique simple d'utilisation pour les étudiants de l'Université Paris Nanterre.

\subsection{Objectifs fonctionnels}
Les principaux objectifs fonctionnels sont :
\begin{itemize}
    \item permettre à un étudiant de créer un compte, se connecter et se déconnecter ;
    \item offrir une interface de chat pour poser des questions à l'IA ;
    \item stocker les questions et réponses dans une base de données, avec un historique par utilisateur ;
    \item afficher un historique des questions/réponses, avec le nom d'utilisateur associé ;
    \item permettre à chaque utilisateur d'effacer son propre historique.
\end{itemize}

\subsection{Scénario d'utilisation}
Un étudiant arrive sur la page d'accueil, se connecte à l'aide de son compte, puis accède à la page de chat. 
Il pose une question sur un sujet lié à l'université (par exemple les horaires de la bibliothèque ou la procédure pour obtenir une attestation de scolarité) et reçoit une réponse générée par l'IA. 
Il peut ensuite consulter l'historique de ses questions, voir les réponses associées et, s'il le souhaite, effacer tout son historique.

% ================== 4. BIBLIOTHÈQUES / TECHNOLOGIES ==================
\section{Bibliothèques, outils et technologies}
Les principales technologies utilisées sont :
\begin{itemize}
    \item \textbf{Python} pour la logique serveur ;
    \item \textbf{Flask} comme framework web pour gérer les routes, les sessions et les templates ;
    \item \textbf{SQLite} pour la base de données (tables utilisateurs, questions, réponses) ;
    \item \textbf{HTML / CSS} pour les pages et la mise en forme (navbar, pages d'historique, formulaires) ;
    \item \textbf{API d'IA} (modèle de langage Mistral) pour générer les réponses aux questions des étudiants ;
    \item \textbf{Git et GitHub} pour le suivi de version et le dépôt public du code source.
\end{itemize}

% ================== 5. FONCTIONNEMENT DE L'IA ==================
\section{Fonctionnement de l'IA utilisée}
L'application s'appuie sur une API d'IA de type modèle de langage (LLM), fournie par le service Mistral AI. 
Le backend Flask construit, pour chaque requête, un \emph{prompt} qui contient la nouvelle question de l'utilisateur ainsi qu'un historique de plusieurs échanges récents récupérés dans la base de données pour ce même utilisateur.

Le prompt inclut également des instructions précises demandant au modèle de :
\begin{itemize}
    \item répondre uniquement sur des sujets liés à l'Université Paris Nanterre (inscription, scolarité, vie de campus, services aux étudiants) ;
    \item refuser poliment les questions sans lien avec l'université ;
    \item produire une réponse claire, structurée, en français et sans emoji.
\end{itemize}

La réponse générée par l'IA est ensuite nettoyée (par exemple suppression éventuelle d'emojis et normalisation des espaces) puis enregistrée dans la table \texttt{reponses}, associée à la question correspondante. 
À chaque nouvelle question, l'IA reçoit donc un contexte personnalisé, limité aux échanges précédents de l'utilisateur connecté, ce qui lui donne un comportement de mémoire par utilisateur.

% ================== 6. COMMENT L'IA NOUS A AIDÉS ==================
\section{Comment l'IA nous a aidé dans le projet}
L'IA nous a aidé à plusieurs niveaux au cours de la réalisation du projet.

\subsection{Aide à la conception}
Nous avons utilisé l'IA pour clarifier le cahier des charges, proposer une structure de base de données (tables \texttt{users}, \texttt{questions}, \texttt{reponses}) et définir les principales routes du serveur (\texttt{/login}, \texttt{/signup}, \texttt{/history}, \texttt{/api/chat}, etc.). 
Elle nous a également aidés à réfléchir à la façon de gérer un historique par utilisateur et à la manière de filtrer les anciennes questions dans le prompt.

\subsection{Aide au développement et au débogage}
Pendant le développement, l'IA a servi d'assistant de programmation pour écrire ou compléter certains fragments de code Flask (gestion des sessions, redirections, formulaires) et SQL (jointures entre les tables, requêtes de sélection de l'historique). 
Nous l'avons aussi utilisée pour diagnostiquer des erreurs (par exemple des requêtes SQL incorrectes, des problèmes de \texttt{commit} sur la base, ou des conditions mal écrites dans les templates Jinja).

\subsection{Aide à la rédaction}
Pour le rapport et pour certains textes affichés dans l'interface, l'IA a été utilisée comme outil de reformulation afin d'obtenir des phrases plus claires et mieux structurées. 
Nous sommes restés responsables de la validation de tout le contenu : chaque proposition de l'IA a été relue, adaptée au contexte du cours et testée avant d'être intégrée.

% ================== 7. TRAVAIL RÉALISÉ ==================
\section{Travail réalisé}
Cette section présente les fonctionnalités prévues au départ et indique celles qui ont été réellement implémentées.

\subsection{Fonctionnalités prévues}
\begin{itemize}
    \item Inscription, connexion et déconnexion des utilisateurs ;
    \item Interface de chat avec l'IA ;
    \item Historique global des questions (vue enseignant) ;
    \item Historique personnel par utilisateur, avec possibilité d'effacer son historique ;
    \item Intégration d'informations officielles (scraping ou saisie manuelle) du site de l'université.
\end{itemize}

\subsection{Fonctionnalités réalisées}
Ici vous détaillerez ce qui est effectivement présent dans votre projet, par exemple :
\begin{itemize}
    \item gestion complète des comptes utilisateurs (inscription, connexion, déconnexion) ;
    \item enregistrement des questions et réponses en base, avec association à un utilisateur ;
    \item page d'historique affichant les questions, réponses, date et pseudo ;
    \item filtrage du contexte IA par utilisateur et bouton pour effacer son historique.
\end{itemize}

\subsection{Fonctionnalités non réalisées et raisons}
Indiquez clairement les fonctionnalités non terminées (par exemple scraping automatique du site) et expliquez pourquoi (manque de temps, complexité technique, priorisation des fonctionnalités essentielles, etc.).

\subsection{Répartition réelle du travail}
Décrivez comment le travail s'est réparti entre les deux membres du groupe (backend, frontend, intégration de l'IA, rédaction du rapport, etc.).

% ================== 8. DIFFICULTÉS ==================
\section{Difficultés rencontrées}
Nous avons rencontré plusieurs difficultés techniques : gestion correcte de l'historique par utilisateur dans la base de données, configuration de l'API d'IA, ainsi que quelques problèmes de sessions et de redirections lors de l'implémentation de l'authentification.

Une autre difficulté a été de cadrer l'IA pour qu'elle reste centrée sur les questions universitaires et qu'elle ne fournisse pas de réponses hors sujet ou potentiellement erronées sans avertissement.

% ================== 9. BILAN ==================
\section{Bilan}
\subsection{Conclusion}
Ce projet nous a permis de mettre en pratique les notions vues en cours (développement web avec Flask, gestion de base de données, intégration d'une API externe) tout en découvrant les enjeux spécifiques de l'intégration d'une IA dans une application réelle.

\subsection{Perspectives}
Plusieurs pistes d'amélioration sont possibles : enrichir la base de connaissances en important automatiquement des informations à jour depuis le site de l'université, améliorer l'interface utilisateur, ajouter une interface d'administration pour analyser les questions fréquentes et régler plus finement le comportement de l'IA.

% ================== 10. BIBLIO / WEBO ==================
\newpage
\section{Bibliographie}
\renewcommand{\bibname}{}
\renewcommand{\refname}{}
\begin{thebibliography}{2}
   \bibitem[LAM94]{lam1} L. Lamport, {\it \LaTeX: A Document Preparation System}, Addison-Wesley, 1994.
\end{thebibliography}

\newpage
\section{Webographie}
\begin{thebibliography}{2}
   \bibitem[DOCFLASK]{flaskdoc} \url{https://flask.palletsprojects.com/}
   \bibitem[SQLITE]{sqlite} \url{https://www.sqlite.org/docs.html}
   \bibitem[MISTRAL]{mistral} \url{https://docs.mistral.ai/}
\end{thebibliography}

% ================== 11. ANNEXES ==================
\newpage
\section{Annexes}
\appendix
\makeatletter
\def\@seccntformat#1{Annexe~\csname the#1\endcsname:\quad}
\makeatother

\section{Cahier des charges}
Ici vous pouvez rappeler les exigences initiales du projet (fonctionnalités demandées, contraintes techniques, etc.).

\section{Exemple d'exécution du projet}
Cette annexe doit contenir un exemple complet d'utilisation de l'application, avec des captures d'écran (page d'accueil, connexion, chat avec une question et une réponse, historique, etc.).

\section{Manuel utilisateur}
Expliquez rapidement comment lancer le projet (cloner le dépôt GitHub, créer le \texttt{venv}, installer les dépendances, lancer Flask) et comment utiliser les principales fonctionnalités.

\end{document}
